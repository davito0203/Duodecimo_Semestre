\documentclass[12pt,graphicx,caption,rotating]{article}
\textheight=23cm
\textwidth=17cm
\topmargin=-1cm
\oddsidemargin=0cm
\usepackage[activeacute,spanish]{babel}
\usepackage[utf8]{inputenc}
\usepackage{graphicx}%manejo de graficos
\usepackage{times}
\usepackage{amssymb,amsfonts}
\usepackage[tbtags]{amsmath}
\usepackage{cite}
\usepackage[all]{xy}
\usepackage{subfigure}
\usepackage{wrapfig}
\usepackage[usenames,dvipsnames]{color}
\usepackage{multicol}
\usepackage{cite}
\usepackage{url}
\usepackage[tbtags]{amsmath}
\usepackage{amsmath,amssymb,amsfonts,amsbsy}
\usepackage{bm}
\usepackage{algorithm}
\usepackage{algorithmic}
\usepackage[all]{xy}
\usepackage{authblk}
\usepackage[centerlast, small]{caption}
\usepackage[colorlinks=true, citecolor=blue, linkcolor=blue, urlcolor=blue, breaklinks=true]{hyperref}
\hyphenation{ele-men-tos he-rra-mi-en-ta cons-tru-yen trans-fe-ren-ci-a pro-pu-es-tas si-mu-lar vi-sua-li-za-cion}

\begin{document}
\title{Manual para la instalación de herramientas}

\author[1]{Martinez Hernandez. David Ricardo}
\author[2]{Tobasura Madero. David Leonardo}
\author[3]{Urbano Vallejo. Oscar Andres}
\affil[1]{\href{}{drmartinezhe@unal.edu.co}}
\affil[2]{\href{}{dltobasuram@unal.edu.co}}
\affil[3]{\href{}{oaurbanov@unal.edu.co}}

\date{}
\maketitle

\section{Instalación Herramientas}
\noindent
Para poder instalar las herramientas del sistema embebido es necesario descomprimir el archivo \textit{\textcolor{blue}{buildroot-2013.08.1.tar.bz2}} el cual contiene las herramientas necesarias para desarrollar adecuadamente el proyecto.\\
Para descomprimir dicho archivo se debe ejecutar lo siguiente en la terminal:
\begin{center}
 \fcolorbox{Gray}{White}{\textcolor{red}{tar -xvjf buildroot.tar.bz2}} 
\end{center}
\noindent
Si aparece algún error al ejecutar dicho comando es necesario hacerlo en modo \textbf{super usuario} o:
\begin{center}
 \fcolorbox{Gray}{White}{\textcolor{red}{sudo tar -xvjf buildroot.tar.bz2}}
\end{center}
\noindent
Para instalar las herramientas que se encuentran en ese archivo se utiliza el siguiente comando:
\begin{center}
 \fcolorbox{Gray}{White}{\textcolor{red}{make}} o \fcolorbox{Gray}{White}{\textcolor{red}{sudo make}}
\end{center}

\bibliographystyle{ieeetran}
\begin{thebibliography}{99}

\bibitem{page1} Referenicas Librerias MIDI \url{http://www.faqs.org/docs/Linux-HOWTO/MIDI-HOWTO.html}

\bibitem{page2} Referenicas Librerias MIDI \url{https://ccrma.stanford.edu/~craig/articles/linuxmidi/}

\bibitem{page3} Referenicas Librerias MIDI \url{http://linux-sound.org/midi.html}

\end{thebibliography}
\end{document}